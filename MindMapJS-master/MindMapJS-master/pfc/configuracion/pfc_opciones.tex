\usepackage[utf8]{inputenc}
\usepackage{geometry}
\geometry{verbose,a4paper,tmargin=2.5cm,bmargin=2.5cm,lmargin=2.5cm,rmargin=2.5cm}
\usepackage{fancyhdr}
\pagestyle{fancy}
\setcounter{tocdepth}{3}
\setlength{\parskip}{\medskipamount}
\setlength{\parindent}{0pt}
\usepackage{float}
\usepackage{framed}
\usepackage{amsthm}
\usepackage{amsmath}
\usepackage{graphicx}
\usepackage{wrapfig}
\usepackage{longtable}
\usepackage{color}
\usepackage[bookmarks]{hyperref}
\usepackage{listings}
\usepackage{listingsutf8}

\pretolerance=2000
\tolerance=3000
%%%%%%%%%%%%%%%%%%%%%%%%%%%%%% Textclass specific LaTeX commands.
\theoremstyle{plain}

%%%%%%%%%%%%%%%%%%%%%%%%%%%%%% User specified LaTeX commands.
\usepackage{fancyhdr}
    \pagestyle{fancy}
    \fancyhf{}
    \fancyhead[LO]{\leftmark} % En las páginas impares, parte izquierda del encabezado, aparecer el nombre de capítulo
    \fancyhead[RE]{\rightmark} % En las páginas pares, parte derecha del encabezado, aparecer el nombre de sección
    \fancyhead[RO,LE]{\thepage} % Nmeros de página en las esquinas de los encabezados
    \fancyfoot[RO]{José Luis Molina Soria}
    \fancyfoot[LO]{Editor de mapas mentales online con HTML5 y Javascript}
    \renewcommand{\chaptermark}[1]{\markboth{\textbf{\thechapter. #1}}{}} % Formato para el capítulo: N. Nombre
    \renewcommand{\sectionmark}[1]{\markright{\textbf{\thesection. #1}}} % Formato para la sección: N.M. Nombre
    \renewcommand{\headrulewidth}{0.6pt} % Ancho de la lnea horizontal bajo el encabezado
    \renewcommand{\footrulewidth}{0.6pt} % Ancho de la lnea horizontal sobre el pie (que en este ejemplo est vaco)
    \setlength{\headheight}{1.5\headheight} % Aumenta la altura del encabezado en una vez y media

\usepackage{babel}
\addto\shorthandsspanish{\spanishdeactivate{~<>}}

\renewcommand{\chaptername}{Parte} 
\renewcommand{\partname}{}
\renewcommand{\bibname}{BIBLIOGRAFÍA}


\makeatletter

\def\thickhrulefill{\leavevmode \leaders \hrule height 1ex \hfill \kern \z@}
\def\@makechapterhead#1{%
  \reset@font
  \parindent \z@ 
  \vspace*{10\p@}%
  \hbox{%
    \vbox{\hsize=2cm
      \begin{tabular}{c}
        \scshape \strut \@chapapp{} \\
        \fbox{%
          \vrule depth 10em width 0pt%
          \vrule height 0pt depth 0pt width 1ex%
          {\LARGE \bfseries \strut \thechapter}%
          \vrule height 0pt depth 0pt width 1ex%
          }
      \end{tabular}%
      }%
    \vbox{%
      \advance\hsize by -2cm
      \hrule\par
      \vskip 6pt%
      \hspace{0em}%
      \Huge \bfseries #1
      }%
    }%
  \vskip 100\p@
}
\def\@makeschapterhead#1{%
  \reset@font
  \parindent \z@ 
  \vspace*{10\p@}%
 \hbox{%
    \vbox{\hsize=2cm
      \begin{tabular}{c}
        \scshape \strut \vphantom{\@chapapp{}} \hphantom{\@chapapp{}} \\
        \fbox{%
          \vrule depth 10em width 0pt%
          \vrule height 0pt depth 0pt width 1ex%
          {\LARGE \bfseries \strut \hphantom{\thechapter}}%
          \vrule height 0pt depth 0pt width 1ex%
          }
      \end{tabular}%
      }%
    \vbox{%
      \advance\hsize by -2cm    
      \hrule\par
      \vskip 6pt%
      \hspace{1em}%
      \Huge \bfseries #1
      }%
    }%
  \vskip 100\p@
}


\usepackage{tocloft}
\renewcommand{\cftsecnumwidth}{3em}
%\renewcommand{\cftsecfont}{3em}
\renewcommand{\cftfignumwidth}{3em}
\setlength{\cftbeforesecskip}{0.7em \@plus\p@}
\setlength{\cftbeforefigskip}{0.4em \@plus\p@}
\renewcommand{\baselinestretch}{1.5}

\definecolor{lightgray}{rgb}{.9,.9,.9}
\definecolor{darkgray}{rgb}{.4,.4,.4}
\definecolor{purple}{rgb}{0.65, 0.12, 0.82}

\lstdefinelanguage{JavaScript}{
  keywords={typeof, new, true, false, catch, function, return, null, catch, switch, var, if, in, while, do, else, case, break},
  keywordstyle=\color{blue}\bfseries,
  ndkeywords={class, export, boolean, throw, implements, import, this},
  ndkeywordstyle=\color{darkgray}\bfseries,
  identifierstyle=\color{black},
  sensitive=false,
  comment=[l]{//},
  morecomment=[s]{/*}{*/},
  commentstyle=\color{purple}\ttfamily,
  stringstyle=\color{red}\ttfamily,
  morestring=[b]',
  morestring=[b]"
}

\lstset{
   language=JavaScript,
   backgroundcolor=\color{lightgray},
   extendedchars=true,
   inputencoding=ansinew,
   basicstyle=\linespread{0.90}\footnotesize\ttfamily,
   showstringspaces=false,
   showspaces=false,
   numbers=left,
   numberstyle=\footnotesize,
   numbersep=9pt,
   tabsize=2,
   breaklines=true,
   showtabs=false,
   captionpos=b
}