\documentclass[14pt]{extreport}
\usepackage[latin1]{inputenc}
\usepackage{geometry}
\geometry{verbose,a4paper}

\usepackage{amsmath}
\usepackage{amsfonts}
\usepackage{amssymb}
\usepackage{graphicx}

\usepackage{fancyhdr}
\pagestyle{fancy}
\usepackage{float}
\usepackage{framed}
\usepackage{amsthm}
\usepackage{longtable}
\usepackage{color}

\begin{document}

\\[2\baselineskip]
\section*{Abstract.}
\\[2\baselineskip]

El editor de mapas mentales on-line es un sistema web desarrollado �nica y exclusivamente en HTML5
para dise�ar y elaborar mapas mentales con formato FreeMind. Formato el cual, es considerado un
est�ndar en el mundo de los mapas mentales.

La idea que subyace en la realizaci�n de este editor de mapas mentales, es probar las nuevas
tecnolog�as existentes alrededor de HTML5 y comprobar el estado actual de dicha
tecnolog�a tras el inter�s y la relevancia que est� adquiriendo en estos �ltimos a�os. 

La metodolog�a �gil, utilizada para llevar a cabo el proyecto, se adapta muy bien al desarrollo web. Permitiendo un feedback constante desde la versi�n inicial hasta la actual. Conjuntamente con la metodolog�a �gil se utilizado otras metodolog�as y paradigmas de programaci�n como el BBD\footnote{Como sistema de pruebas y verificaci�n del c�digo fuente}, patrones de dise�o, etc ... Tambi�n se ha hecho uso de tecnolog�as ya existentes como soporte al desarrollo, entre ellas destacar KineticJS, Mocha, GruntJS, NodeJS, JSHint, JSDoc y GitHub. Todas estas tecnolog�as han propiciado una experiencia de desarrollo satisfactoria. 

En resumen, se ha podido constatar un gran avance en HTML5 con respecto a la versi�n anterior. A pesar de ello, sigue existiendo, aunque en mucho menor grado, una gran dependencia con el navegador. 
\end{document}